
%\documentclass[preprint,superscriptaddress,showkeys]{revtex4}
\documentclass[twocolumn,superscriptaddress,showkeys]{revtex4}

\usepackage{amssymb}

\begin{document}


\title{Phylogenetic structure of an ecological restoration through time.}

\author{William A. Freyman}
\email{freyman@berkeley.edu}
\affiliation{Department of Integrative Biology, University of California, Berkeley}
%\date{}

\begin{abstract}
Blah blah
\end{abstract}

\keywords{phylogenetic diversity, ecological restoration, conservation planning, native species, evolutionary history}

\maketitle


\section{Introduction}


Phylogenetic diversity as a goal of restoration ecology \cite{Hipp2015, Winter2013, rosauer2013, Verdu2012, Vane-Wright1991}.
Conservation goals include preserving evolutionary history and genetic diversity...tree of life


The problem with species based metrics compared to PD. PD as proxy to evolutionary potential and functional diversity.


PD could be highly useful for conservation planning and assess restoration for these reasons:
1: how do patterns of phylogenetic structure in plant communities change through time as they are restored?
2: quantifying thje amount of evolutionary history that has accumalated in a certain ecosystem and can be restored
3: what is the effect of non-native species on phylogenetic structure
4: PD metrics can be more objective than metrics like FQA that use subjective coefficients. 
PD can incorporate phylo uncertainity by calculating PD over a credible interval, typically 95\% Highest Posterior Density (HPD).
FQA represent a point estimate, and does not include an of the uncertainity that went into the subjective assignment of coefficients of conservativism.


Futhermore, there are two type of PD that are useful: genetic diversity and evolutionary history (in millions of years).
describe.


If PD is so useful, what has it not been used more? explain.
different goals and perspectives within the fields of applied ecology and evolution.
technical difficulty of making PD accessible to applied ecologists.
PD must be made as accessible as FQA is for example Universal Floristic Quality Assessment Calculator \cite{Freyman2013}


Goals of this paper:
1: empirically observe how patterns of phylogenetic structure in plant communities change through time as they are restored.
2: quantifying thje amount of evolutionary history
3: measure the effect of non-native species on phylogenetic structure
4: compare PD to FQA
5: assess the impact of phylogenetic uncertainity on PD metrics


\section{Methods}

Describe study site Somme Prairie Grove and Vestal Grove transect sampling.

Phylogeny of 2215 vascular plants

Taxonomy followed \cite{Herman2014} GenBank mining (GenBank accessions: (Appendix S1)), building uncalibrated phylogeny.


Time calibrating phylogeny.


Calculating PD metrics. Should try cover weighting? Calculate 95\% HPD


Calculate FQAs \cite{Freyman2013}.


\section{Results}

Describe uncalibrated and calibrated phylogenies. Include table of gene regions and their sampling.


Describe PD in time and genetic diversity over time. Describe cover-weighted PD.


Describe changes in \% non-native PD.


Describe changes in FQA over time.


Figures: \\
1) uncalibrated phylogeny w/ natives non-natives colored \\
2) calibrated phylogeny \\
3) graph: PD in million of years through time \\
4) graph: PD in nucleotide substitions through time \\
5) graph: native PD in million of years through time \\
6) graph: native PD in nucleotide substitions through time \\
7) graph: non-native PD in million of years through time \\
8) graph: non-native PD in nucleotide substitions through time \\
9) graph: percent non-native PD in million of years through time \\
10) graph: percent non-native PD in nucleotide substitions through time \\
11) graph: cover-weighted native PD in million of years through time \\
12) graph: cover-weighted native PD in nucleotide substitions through time \\
13) graph: cover-weighted non-native PD in million of years through time \\
14) graph: cover-weighted non-native PD in nucleotide substitions through time \\
15) graph: cover-weighted percent non-native PD in million of years through time \\
16) graph: cover-weighted percent non-native PD in nucleotide substitions through time \\
17) graph: FQA through time

\section{Discussion}


Discuss overall trends: native/non-native and PD/FQA


Discuss clade specific trends.


Discuss impact of phylogenetic uncertainity on results. Are metrics of PD robust to phylogenetic uncertainity?


Discuss other ways of looking at data, for example transect across ecological gradient, 
the phylogenetic turnover through the transition zone of adjoining prairie and woodland ecosystems.


Discuss range-weighted phylogenetic endemism metrics and how they might be useful.


How to make PD accessible to applied ecologists? Discuss
modifying the Universal Floristic Quality Assessment Calculator \cite{Freyman2013}
to include PD.


state conclusions.
Applying PD metrics can be useful for all conservation planning, not only ecological restorations.

\section{Supporting Information}

The GenBank accessions used (Appendix S1) are available online as Supplementary Material. 

\section{Competing interests}

We have no competing interests.

\section{Authors' contributions}

WAF conceived of the study, performed the phylogenetic analysis, and drafted the manuscript. 
SP and LAM collected field data. All authors gave final approval for publication.

\section{Acknowledgements}

\section{Funding}

\bibliography{vestal}


\end{document}

